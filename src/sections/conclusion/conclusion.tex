\documentclass[../../../main]{subfiles}
\begin{document}

\section{結論}
今回の実験では、MTJ素子におけるTMR効果を測定しTMR比が\SI{153.8}{\%}と求まった。
また、スピン分極率は\SI{65.90}{\%}と求まった。
このことから、素子がほぼ同程度ずつアップとダウンのスピンを持っていることがわかった。
また、Julliereモデルを元に磁化と電気抵抗を考え、
TMR比を元に常磁性体に外部磁場を加えた時の残留磁化、ヒステリシスを考察した。
さらに、電子のスピンのバンド図について統計力学的視点から考察した。

今後の課題として、今回使用したMTJ素子は磁場の向きに対して特性があり、
今回の正の向きでは抵抗が変化した一方で、負の場合には変化が見られなかった。
この特性がなぜ生まれるのか、についてより深く調査する必要がある。


\end{document}
